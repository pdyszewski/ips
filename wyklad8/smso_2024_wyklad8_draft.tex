\documentclass{article}

%\usepackage{MyPackage}
\usepackage{Stochastyczne_modele_systemow_oddzialujacych_Package}
\begin{document}
SMUO 2024

wykład 7: Rozkłady stacjonarne, zaburzenia ruchu Browna

\section{Motywacje}

Aby umotywować nasze przyszłe działania przedyskutujemy kilka przykładów. 
Od tej pory niech $G =(V, E)$ będzie przeliczalnym grafem prostym o ograniczonym stopniu.
Dokładniej zakładać będziemy, że zbiór jego wierzchołków $V$ jest skończony bądź przeliczalny 
oraz, że
\begin{equation*}
	\sup_{x \in V} \mathrm{deg}(x) <\infty.
\end{equation*}
Rozważmy następujące trzy procesy na $G$.

\begin{pd}[Voter model]
	Przypuśćmy, że na $G$ są dwie wzajemnie zwalczające się frakcje (można myśleć o republikanach 
	i demokratach). Na każdy wierzchołek jednej frakcji oddziałują (poprzez indoktrynację) 
	sąsiedzi z frakcji przeciwnej. Na skutek czego niektóre wierzchołki zmieniają frakcję
	na przeciwną.
	Dokładniej każdy $x \in V$ zmienia frakcję w intensywnością równą
	\begin{equation*}
		c(x) = \# \{ y \in V \: \: x \sim y, \: \mbox{$y$ jest z innej frakcji niż $x$} \}.
	\end{equation*}
	Jest to równoważne z następującym opisem.
	\begin{itemize}
		\item Dla każdej pary sąsiednich wierzchołków $x\sim y$ pochodzących z różnych 
			frakcji losujemy niezależnie dwie liczby $E_{x\to y}$ oraz $E_{y\to x}$
			z rozkładu wykładniczego z parametrem jeden. $E_{x\to y}$ interpretujemy
			jako czas, jaki potrzebuje $x$ aby przekabacić $y$.
		\item W momencie, w którym którykolwiek z wierzchołków przeciągnie sąsiada 
			(powiedzmy $y$)
			na swoją stronę, losujemy nowe wagi dla sąsiadów $y$ (według nowego 
			układu obu frakcji).
	\end{itemize}
	Oczywiście powyższa, naiwna konstrukcja ma sens tylko, gdy graf $G$ jest skończony. 
	W przypadku grafów nieskończonych wymagana jest odpowiednia konstrukcja, którą opiszemy
	niebawem.
\end{pd}

\begin{pd}[Contact process]
	Załóżmy, że na grafie $G$ panuje epidemia. Niech $\lambda >0$ będzie ustalonym parametrem.
	Proces rozwija się według następujących zasad.
	\begin{itemize}
		\item Chore wierzchołki niezależnie 
			zarażają swoich zdrowych sąsiadów z czasem wykładniczym 
			z parametrem $\lambda$.
		\item Chore wierzchołki zdrowieją niezależnie z czasem 
			wykładniczym z parametrem jeden.
	\end{itemize}
	Ponownie powyższy opis ma sens jedynie w przypadku, gdy $G$ jest skończony.
	Dla nieskończonych grafów $G$ będziemy posługiwać się równoważną charakteryzacją
	\begin{itemize}
		\item Każdy chory wierzchołek zdrowieje z intensywnością $1$
		\item Każdy zdrowy wierzchołek $x$ choruje z intensywnością 
			\begin{equation*}
				\lambda \# \{ y\sim x \: : \: y \mbox{ chory}\}.
			\end{equation*}
	\end{itemize}
\end{pd}

\begin{pd}[Exclusion process]
	Rozważmy zjawisko migracji na grafie $G$. Załóżmy, że pewne wierzchołki $G$ są zajmowane 
	przez osobników ustalonego gatunku (przykładowo krowy). Każdy osobnik
	czeka niezależnie wykładniczy czas z parametrem jeden, po czym podejmuje próbę 
	przemieszczenia się. 
	\begin{itemize}
		\item Jeżeli osobnik zajmuje wierzchołek $x \in V$, to losuje wierzchołek $y$ 
			z prawdopodobieństwem $p(x,y)$.
		\item Jeżeli wierzchołek $y$ nie jest zajęty, to osobnik przechodzi do $y$.	
	\end{itemize}
\end{pd}

W każdym z powyższych przykładów stan układu w chwili $t$ można zakodować przy pomocy 
$\eta_t \in \{ 0,1\}^V$. Okazuje się, że wówczas $\eta = (\eta_t)_{t \in \mathbb{R}_+}$
staje się procesem Fellera. Aby się o tym dokładnie przekonać musimy przeanalizować
generatory wynikające z przykładów.

\section{Systemy spinowe}

Właściwością, która odróżnia systemy spinowe od innych procesów Fellera na $\{0,1\}^S$, 
jest to, że poszczególne przejścia obejmują tylko jedną lokalizację. 
Niech $c \colon V \times \{0,1\}^V \to \mathbb{R}_+$ będzie ograniczoną 
funkcją taką, że dla każdego $x \in V$, $c(x, \cdot) \colon \{0,1\}^V \to \mathbb{R}_+$ jest
ciągła. $c(x,\eta)$ interpretować będziemy jako intensywność, z jaką stan $\eta$ zmienia się poprzez
zmianę wartości w $x$. Dla $x \in V$ oraz $\eta \in \{0,1\}^V$ definiujemy $\eta^{(x)} \in \{0,1\}^V$
wzorem
\begin{equation*}
	\eta^{(x)}(y) = \left\{ 
	\begin{array}{cc} \eta(y), & y \neq x \\ 1-\eta(x), & y=x \end{array}\right..
\end{equation*}
Dla $f$ pochodzącego z odpowiedniego podzbioru  $C_0(\{0,1\}^V)$ chcemy
położyć
\begin{equation}\label{eq:4:defL}
	Lf(\eta) = \sum_{x\in V} c(x, \eta)\left[f\left(\eta^{(x)}\right) - f(\eta)\right].
\end{equation}
Okazuje się, że dokładne napisanie dziedziny jest problematyczne. Aby obejść tę trudność 
rozważmy
\begin{equation}\label{eq:4:defD}
	D = \left\{ f \in C(\{0,1\}^V) : \|f\|_o := \sup_{\eta} \sum_x \left|f\left(\eta^{(x)}\right) - f(\eta)\right| < \infty \right\}.
\end{equation}

Naszym celem jest pokazanie, że zdefiniowanie $L$ na $D$ wystarcza do zdefiniowania procesu.

\section{Dygresja analityczna}

	Generator infinitezymalny nie należy do najprostszych obiektów w teorii procesów Fellera.
	Jednym z powodów jest konieczność uwzględnienia dzieciny która, jak się już przekonaliśmy,
	ma istotny wpływ na kształt generowanego procesu.
	Rzadko się jednak zdarza, że dziedzinę można opisać jawnie.
	Dlatego często definiuje się proponowany generator na wygodnej
	podprzestrzeni dziedziny, a następnie bierze się domknięcie.

	\begin{defn}
		Operator liniowy $(L, \mathcal{D}(L))$ na \( C_0(S) \)
		nazywany jest domkniętym, jeśli jego wykres
		\[
			\Gamma(L)=\{(f, Lf) : f \in D(L)\}
		\]
		jest domkniętym podzbiorem \( C_0(S) \times C_0(S) \).
	\end{defn}

	Operator $L$ jest domknięty wtedy i tylko wtedy, gdy $f_n \in \mathcal{D}(L)$ są takie, że
	$f_n \to f$ oraz $Lf_n \to h$, to $f \in \mathcal{D}(L)$ oraz $h=Lf$.

	\begin{defn}
		Operator liniowy $(L, \mathcal{D}(L))$ nazywany jest domykalnym, jeżeli
		domknięcie jego wykresu $\Gamma(L)$ jest wykresem operatora liniowego. W takiej
		sytuacji  definiujemy domknięcie $\overline{L}$ operatora $L$
		poprzez
		\begin{equation*}
			\Gamma \left( \overline{L} \right) = \overline{\Gamma(L)}.
		\end{equation*}
	\end{defn}

	Operator $L$ jest domykalny wtedy i tylko wtedy, gdy $f_n \to 0$ oraz $Lf_n \to h$ implikują
	$h=0$. Nie każdy operator liniowy ma domknięcie. Na przykład, przypuśćmy, że \( S = [0, 1] \) i
	\[
		\mathcal{D}(L) = \{ f \in C(S) : f'(0) \text{ istnieje} \}
		\quad \text{i} \quad Lf(x) = f'(0)x \text{ dla } f \in \mathcal{D}(L).
	\]
	Wtedy domknięcie wykresu \( L \) nie jest wykresem operatora liniowego.
	Jednakże w kontekście Definicji~\ref{defn:3:12} taka sytuacja nie występuje.

	\begin{fak}\label{fak:3:30}
		Niech $(L, \mathcal{D}(L))$ będzie operatorem liniowym na $C_0(S)$.
		\begin{itemize}
    			\item[(a)] Przypuśćmy, że \( L \) spełnia (GI1)-(GI3) 
				z Definicji~\ref{defn:3:12}.
				Wtedy $L$ jest domykalny, a jego domknięcie spełnia (GI1)-(GI3).
    			\item[(b)] Jeśli \( L \) spełnia (GI1)- (GI4) z Definicji~\ref{defn:3:12},
				wtedy \( L \) jest domknięty.
			\item[(c)] Jeśli $L$ 
				spełnia (GI3) i (GI4), to
    				\[
    					\mathcal{R}(I - \lambda L) 
					= C_0(S) \quad \text{dla każdego } \lambda > 0.
    				\]
    			\item[(d)] Jeśli $L$ jest domknięty i spełnia (GI3), to 
				$\mathcal{R}(I - \lambda L)$ 
				jest domkniętym podzbiorem $C_0(S)$.
		\end{itemize}
	\end{fak}

	\begin{proof}
		Dla pierwszego stwierdzenia, musimy udowodnić, że
		\( f_n \in \mathcal{D}(L), f_n \to 0 \),
		oraz \( Lf_n \to h \) implikuje \( h = 0 \).
		Aby to zrobić, wybierzmy \( g \in \mathcal{D}(L) \).
		Korzystając z \eqref{eq:3:12},
		\begin{equation*}
			\|(I - \lambda L)(f_n + \lambda g)\|
			\geq \|f_n + \lambda g\|,
			\quad \lambda > 0.
		\end{equation*}
		Biorąc \( n \to \infty \) i następnie dzieląc przez \( \lambda \), dostajemy
		\[
			\|g - h - \lambda Lg\| \geq \|g\|.
		\]
		Jeżeli teraz wybierzemy \( \lambda \to 0 \), to otrzymamy
		\begin{equation*}
			\| g-h \| \geq \|g\|.
		\end{equation*}
		Skoro $g\in \mathcal{D}(L)$ jest wzięte z gęstego zbioru otrzymujemy
		\( h = 0 \). Domknięcie \( \overline{L} \) spełnia własności (GI1) i (GI2),
		ponieważ jest rozszerzeniem \( L \). Aby sprawdzić, czy spełnia własność (GI3),
		przypuśćmy, że \( f \in \mathcal{D}(\overline{L}), \lambda \geq 0 \) i
		$f - \lambda \overline{L}f = g$. 
		Przez definicję domknięcia, istnieją $f_n \in \mathcal{D}(L)$,
		takie że $f_n \to f$ i $Lf_n \to \overline{L}f$. Przez własność (GI3) dla $L$,
		\[
			\inf_{x \in S} f_n(x) \geq \inf_{x \in S} g_n(x),
		\]
		gdzie $g_n = f_n - \lambda Lf_n$. Teraz niech $n \to \infty$. Dostajemy
		\[
			\inf_{x \in S} f(x) \geq \inf_{x \in S} g(x),
		\]
		czyli własność $(GI3)$ dla $\overline{L}$.

		Dla dowodu drugiej części faktu, niech $\overline{L}$ będzie domknięciem $L$.
		Jeśli $f \in \mathcal{D}(\overline{L})$ i $\lambda > 0$ jest małe, przez własność (GI4)
		istnieje $h \in \mathcal{D}(L)$ takie że
		\begin{equation}\label{eq:3:25}
			h - \lambda Lh = f - \lambda \overline{L}f,
		\end{equation}
		czyli $(h - f) - \lambda \overline{L}(h - f) = 0$.
		Z \eqref{eq:3:12}, $h = f$, a wtedy $\overline{L}f = Lh$ z \eqref{eq:3:25}.
		Więc, $\overline{L} = L$ zgodnie z tezą.

	Przechodząc do trzeciego stwierdzenia, 
	wystarczy sprawdzić, że dla $0 < \lambda < \gamma$ warunek 
	$\mathcal{R}(I - \lambda L) = C_0(S)$  implikuje $\mathcal{R}(I - \gamma L) = C_0(S)$. 
	Załóżmy, że $g \in C_0(S)$, i zdefiniujmy $\Gamma : C(S) \to \mathcal{D}(L)$ przez
	\[
		\gamma \Gamma h 
		= \lambda (I - \lambda L)^{-1} g + (\gamma - \lambda)(I - \lambda L)^{-1} h.
	\]
	Definicja ta jest poprawna, ponieważ założyliśmy 
	$\mathcal{R}(I - \lambda L) = C_0(S)$. Mamy
	\[
		\gamma \| \Gamma h_1 - \Gamma h_2 \| 
		= (\gamma - \lambda) \| (I - \mathcal{L})^{-1} (h_1 - h_2) \| 
		\leq (\gamma - \lambda) \| h_1 - h_2 \|.
	\]
	Stąd $\Gamma$ jest odwzorowaniem zwężającym, a więc z Twierdzenia Banacha o punkcie stałym 
	posiada jedyny punkt stały $f$. 
	Wówczas $f \in \mathcal{D}(L)$ oraz
	\[
		\gamma (I - \lambda \mathcal{L}) f 
		= \lambda g + (\gamma - \lambda) f.
	\]
	Co można przekształcić do postaci $f - \gamma L f = g$.
	Czyli $g \in \mathcal{R}(I - \gamma L)$, co należało uzasadnić.

	Aby udowodnić ostatnie stwierdzenie, załóżmy, że 
	$g_n \in \mathcal{R}(I - \lambda L)$ i $g_n \to g$. 
	Wtedy możemy zdefiniować $f_n \in \mathcal{D}(L)$ przez
	\begin{equation}\label{eq:3.26}
		f_n - \lambda L f_n = g_n.
	\end{equation}
	Wówczas
	\[
		(f_n - f_m) - \lambda L (f_n - f_m) = g_n - g_m,
	\]
	a zatem $\| f_n - f_m \| \leq \| g_n - g_m \|$. 
	Ponieważ $\{g_n\}_{n \in \mathbb{N}}$ jest ciągiem Cauchy’ego, 
	to $\{f_n\}_{n \in \mathbb{N}}$ również. 
	Niech $f = \lim_{n \to \infty} f_n$. 
	Ponieważ $f_n \to f$ i $g_n \to g$, to z \eqref{eq:3.26} wynika, 
	że $\lim_{n \to \infty} L f_n$ również istnieje. 
	Ponieważ $L$ jest domknięte, granicą jest $L f$, a więc $f - \lambda L f = g$, 
	co oznacza, że $g \in \mathcal{R}(I - \lambda L)$, czego należało dowieść.
\end{proof}




\section{Konstrukcja systemów spinowych}

	Naszym pierwszym celem jest znalezienie naturalnych warunków na $c(x, \eta)$, 
	które gwarantują, że domknięcie $(L,D)$, gdzie $L$ i $D$ są dane odpowiednio
	przez~\eqref{eq:4:defL} oraz~\eqref{eq:4:defD} jest operatorem infinitezymalnym. 
	Warunki (GI1), (GI2), (GI3) i (GI5) są łatwe do sprawdzenia i nie wymagają dodatkowych 
	założeń. Prawdziwym wyzwaniem jest (GI4).

	Dla warunku (GI2), używamy twierdzenia Stone’a-Weierstrassa: 
	$D$ jest algebrą funkcji ciągłych na zbiorze zwartym, 
	która rozdziela punkty. Istotnie, dla $\eta \neq \zeta$ istnieje $x \in S$ takie, 
	że $\eta(x) \neq \zeta(x)$. Funkcja $f(\xi) = \xi(x)$ rozdziela 
	$\eta$ i $\zeta$. Algebra $D$ zawiera funkcje stałe, 
	więc $\overline{D} = C(\{0,1\}^V)$. 

	Dla (GI3), załóżmy $f \in D$, $\lambda \geq 0$, oraz $f - \lambda Lf = g$. 
	Ponieważ $\{0,1\}^V$ jest zbiorem zwartym i $f$ jest ciągła, 
	istnieje takie $\eta$, dla którego $f$ osiąga swoje minimum. 
	Wtedy $Lf(\eta) \geq 0$, więc
	\[
		\min_\zeta f(\zeta) 
		= f(\eta) \geq g(\eta) 
		\geq \min_\zeta g(\zeta).
	\]

	Warunek (GI5) wynika z faktu, że $1 \in D$ oraz $L1 = 0$.

	Aby sprawdzić (GI4) musimy wyprowadzić ograniczenie 
	dla rozwiązań równania $f - \lambda Lf = g$. Niech
	\[
		\epsilon = \inf_{u, \eta} [c(u, \eta) + c(u, \eta_u)] \quad \text{oraz} 
		\quad \gamma(x, u) = \sup_\eta |c(x, \eta_u) - c(x, \eta)|.
	\]
	Zauważmy, że $\gamma(x,u)$ mierzy stopień, w jakim intensywność zmiany w miejscu $x$ 
	zależy od konfiguracji w miejscu $u$. 
	Niech $\ell_1(V)$ będzie przestrzenią Banacha 
	funkcji $\alpha : V \to \mathbb{R}$, które spełniają
	\[
		||\alpha|| := \sum_x |\alpha(x)| < \infty.
	\]
	Macierz $\gamma$ definiuje operator $\Gamma$ na $\ell_1(S)$ przez
	\[
		\Gamma \alpha(u) = \sum_{x: x \neq u} \alpha(x) \gamma(x, u).
	\]
	Operator ten jest dobrze zdefiniowany i ograniczony, pod warunkiem że
	\[
		M := \sup_x \sum_{u: u \neq x} \gamma(x, u) < \infty,
	\]
	a wtedy $||\Gamma|| = M$.


	Dla $f \in C(\{0,1\}^V)$ i $x \in S$, niech
	\[
		\Delta f(x) = \sup_{\eta} \left|f\left(\eta^{(x)}\right) - f(\eta)\right|.
	\]
	Wtedy $\|f\|_o = ||\Delta f||_{l_1(V)}$. Oto oszacowanie, którego potrzebujemy.

	\begin{fak}\label{fak:4.2}
		Załóżmy, że spełniony jest jeden z warunków
		\begin{itemize}
    			\item[(a)] $f \in D$,
    			\item[(b)] $f$ jest ciągła i
				\begin{equation}\label{eq:4.3}
			c(x, \cdot) \equiv 0 
			\text{ dla wszystkich oprócz skończonej liczby } x \in V.
		\end{equation}
		\end{itemize}
	Wówczas jeśli $f - \lambda Lf = g \in D$, $\lambda > 0$, 
	oraz $\lambda M < 1 + \lambda \epsilon$, to
	\begin{equation}\label{eq:4.4}
		\Delta f \leq \left[ (1 + \lambda \epsilon)I - \lambda \Gamma \right]^{-1} \Delta g,
	\end{equation}
	gdzie nierówność zachodzi współrzędna po współrzędnej, a odwrotność 
	jest zdefiniowana przez nieskończony szereg
	\begin{equation}\label{eq:4.5}
		\left[ (1 + \lambda \epsilon)I - \lambda \Gamma \right]^{-1} \alpha 
		= \frac{1}{1 + \lambda \epsilon} \sum_{k=0}^{\infty} 
		\left( \frac{\lambda}{1 + \lambda \epsilon} \right)^k \Gamma^k \alpha.
	\end{equation}
\end{fak}
\begin{proof}
	Zauważmy, że szereg w \eqref{eq:4.5} jest zbieżny dla $\alpha \in \ell_1(V)$ 
	na mocy założenia $\lambda M < 1 + \lambda \epsilon$. 
	Pisząc $f - \lambda Lf = g$ w punktach $\eta$ oraz $\eta^{(u)}$, odejmując i
	zauważając że $(\eta^{(u)})^{(u)} = \eta$, otrzymujemy
	\begin{multline}\label{eq:4.6}
		[f(\eta^{(u)}) - f(\eta)][1 + \lambda c(u, \eta) + \lambda c(u, \eta^{(u)})] 
		= [g(\eta^{(u)}) - g(\eta)]\\
		+ \lambda \sum_{x:x \neq u} \left\{ c(x, \eta^{(u)}) [f((\eta^{(u)})^{(x)}) 
		- f(\eta^{(u)})] - c(x, \eta)[f(\eta^{(x)}) - f(\eta)] \right\}.
	\end{multline}
	Ponieważ wartości $f(\eta^{(u)}) - f(\eta)$, gdy $\eta$ zmienia się a $u$ jest ustalone, 
	tworzą zbiór symetryczny, a ta różnica jest funkcją ciągłą $\eta$, 
	dla każdego $u$ istnieje takie $\eta$, że
	\[
		f(\eta^{(u)}) - f(\eta) = \sup_{\zeta} |f(\zeta^{(u)}) - f(\zeta)| = \Delta f(u).
	\]
	Stąd,
	\[
		f(\zeta^{(u)}) - f(\zeta) \leq f(\eta^{(u)}) - f(\eta)
	\]
	dla każdej $\zeta$. Stosując to dla $\zeta = \eta^{(x)}$ i przekształcając, otrzymujemy
	\[
		f((\eta^{(u)})^{(x)}) - f(\eta^{(u)}) 
		= f((\eta^{(x)})^{(u)}) - f(\eta^{(u)}) \leq f(\eta^{(x)}) - f(\eta),
	\]
	Używając tej nierówności w \eqref{eq:4.6},
	\begin{multline}\label{eq:4.7}
		\Delta f(u)(1 + \lambda \epsilon) \leq 
		\Delta f(u)[1 + \lambda c(u, \eta) + \lambda c(u, \eta^{(u)})] \\
		\leq \Delta g(u) + \lambda \sum_{x:x \neq u} 
		\left[ c(x, \eta^{(u)}) - c(x, \eta) \right] [f(\eta^{(x)}) - f(\eta)]
		\\ \leq \Delta g(u) + \lambda \sum_{x:x \neq u} \gamma(x,u) \Delta f(x).
	\end{multline}
	Jeśli \eqref{eq:4.3} zachodzi, to tylko skończona liczba wyrazów po 
	prawej stronie jest niezerowa, więc 
	$\|\Delta_f\|_1 = \|f\|_o <\infty$. 
	Zastem przy któregokolwiek z założeń faktu, 
	$f \in D$. 
	Dlatego \eqref{eq:4.7} można zapisać jako
	\[
		(1 + \lambda \epsilon) \Delta f \leq \Delta g + \lambda \Gamma \Delta f.
	\]
	Iteracja tej nierówności prowadzi to do
	\[
		\Delta f \leq \frac{1}{1 + \lambda \epsilon} 
		\sum_{k=0}^{n} \left( \frac{\lambda}{1 + \lambda \epsilon} \right)^k 
		\Gamma^k \Delta g + \left( \frac{\lambda}{1 + \lambda \epsilon} \right)^{n+1} 
		\Gamma^{n+1} \Delta f.
	\]
	Jeżeli rozważymy teraz $n \to \infty$, dostaniemy \eqref{eq:4.4}.
\end{proof}


\begin{thm} \label{thm:4.3}
	Załóżmy, że $M < \infty$. 
	Wtedy $\overline{L}$ jest generatorem infinitezymalnym półgrupy 
	Fellera $T=(T(t))_{t \in \mathbb{R}_+}$. 
	Ponadto,
	\begin{equation}\label{eq:4.8}
		\Delta T(t)f \leq e^{-t \epsilon} e^{t \Gamma} \Delta f.
	\end{equation}
	W szczególności, jeśli $f \in D$, to $T_tf \in D$ oraz
	\begin{equation}\label{eq:4.9}
		\|T(t)f\|_o \leq e^{(M - \epsilon)t} \|f\|.
	\end{equation}
\end{thm}

\begin{proof}
	Własności (GI1), (GI2), (GI3) i (GI5) z Definicji~\ref{defn:3:12} 
	zachodzą dla $(L, D)$ są  i są dziedziczone przez $\overline{L}$ z Faktu~\ref{fak:3:30}. 
	Aby sprawdzić warunek (GI4) weźmy  wstępujący ciąg $V_n\subseteq V$ taki, że
	$\bigcup_nV_n=V$. Niech
	\begin{equation}\label{eq:4.10}
		L_n f(\eta) = 
		\sum_{x \in V_n} c(x, \eta) [f(\eta_x) - f(\eta)], 
		\quad f \in C(\{0,1\}^V).
	\end{equation}
	To jest generator dla systemu spinowego, w którym współrzędne
	\[
		(\eta_t(x) : x \notin V_n)
	\]
	są stałe w czasie. 
	Ponieważ $L_n$ jest ograniczonym generatorem, spełnia
	\[
		\mathcal{R}(I - \lambda L_n) = C(\{0,1\}^V)
	\]
	dla wszystkich $\lambda > 0$. 
	Dla $g \in D$, możemy zdefiniować $f_n \in C(\{0,1\}^V)$ 
	przez $f_n - \lambda L_n f_n = g$. 
	Ponieważ $L_n$ spełnia \eqref{eq:4.3}, jeśli $\lambda$ jest wystarczająco małe, 
	tak że $\lambda M < 1 + \lambda \epsilon$, wtedy $f_n \in D$ zgodnie z Faktem~\ref{fak:4.2}. 
	W związku z tym możemy położyć
	\[
		g_n = f_n - \lambda L f_n \in \mathcal{R}(I - \lambda L).
	\]
	Niech $K = \sup_{x, \eta} c(x, \eta) <\infty$, wtedy z Faktu~\ref{fak:4.2}
	\begin{multline}\label{eq:4.11}
		\|g_n - g\| 
		= \lambda ||(L - L_n) f_n|| 
		\leq \lambda K \sum_{x \notin V_n} \Delta f_n(x)\\
		\leq \lambda K \sum_{x \notin V_n} 
		\left[ (1 + \lambda \epsilon)I - \lambda \Gamma \right]^{-1} \Delta g(x).
	\end{multline}
	Ponieważ $\Delta g \in \ell_1(V)$, prawa strona \eqref{eq:4.11} 
	dąży do zera, gdy $n \to \infty$, więc $g_n \to g$. 
	Stąd $g \in \mathcal{R}(I - \lambda L)$, więc wnioskujemy, 
	że $D \subseteq \mathcal{R}(I - \lambda L)$. 
	Ponieważ $D$ jest gęste w $C(\{0,1\}^V)$, 
	widzimy, że $\mathcal{R}(I - \lambda L)$ jest również gęste. Zatem
	\[
		\mathcal{R}(I - \lambda \overline{L}) = C(\{0,1\}^V)
	\]
	zgodnie z Faktem~\ref{fak:3:30}. 
	To kończy weryfikację, że $\overline{L}$ jest generatorem infinitezymalnym.


	Przechodząc do drugiego stwierdzenia, zapiszmy \eqref{eq:4.4} jako
	\[
		\Delta_{(I - \lambda L)^{-1}} g 
		\leq \left[ (1 + \lambda \epsilon)I - \lambda \Gamma \right]^{-1} \Delta g,
	\]
	a następnie iterujmy, aby uzyskać
	\[
		\Delta_{(I - \frac{t}{n} L)^{-1}} g \leq 
		\left[ \left( 1 + \frac{t}{n} \epsilon \right) I - \frac{t}{n} \Gamma \right]^{-n} 
		\Delta g.
	\]
	Przechodząc do granicy otrzymujemy \eqref{eq:4.8}.
\end{proof}
\end{document}
