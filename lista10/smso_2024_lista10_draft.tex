\documentclass{article}

\usepackage{MyPackage}

\begin{document}
SMUO 2024

lista 10: Model epidemii

\begin{enumerate}
	\item Niech $V$ będzie zbiorem przeliczalnym. Rozważmy system spinowy z generatorem 
		infinitezymalnym postaci
		\begin{equation*}
			Lf(\eta) = 
			\sum_{x \in V} 
			c(x, \eta) \left[ f(\left(\eta^{(x)} \right) - f(\eta) \right],
		\end{equation*}
		gdzie liczby $c(x, \eta)$ dla $x\in V$ oraz $\eta\in \{0,1\}^V$ są takie, że
		\begin{equation*}
			\epsilon = \inf_{x,\eta}\left[c(x,\eta) +c\left(x, \eta^{(x)}\right) \right], 
			\quad
			\gamma(x,y) = \sup_{\eta} \left|c\left(x,\eta^{(x)}\right) -c(x,\eta) \right|
		\end{equation*}
		spełniają
		\begin{equation*}
			M = \sup_{x \in V} \sum_{y \neq x} \gamma(x,y) <\epsilon.
		\end{equation*}
		\begin{itemize}
			\item[a.] Pokaż, że dla każdych $\eta, \xi \in \{ 0,1\}^V$
				i $f \in D$ (patrzy wykład 9) mamy
				$|f(\eta) - f(\xi)| \leq \|f\|_o$. Wywnioskuj, że
				$|T_tf(\eta) - T_tf(\xi)| \leq e^{(M-\epsilon)t}\|f\|_o$,
				gdzie $(T_t)_{t \in \mathbb{R}_+}$ jest stowarzyszoną
				półgrupą Fellera.
			\item[b.] Niech $\pi$ będzie rozkładem stacjonarnym dla rozważanego 
				procesu Fellera $(\eta_t)_{t \in \mathbb{R}_+}$. Pokaż,
				że dla każdej ciągłej i rzeczywistej $f$ na $\{0,1\}^V$
				i dla każdej miary probabilistycznej $\nu$ na $\{0,1\}^V$
				mamy
				\begin{equation*}
					\lim_{t \to \infty} \Ebf_\nu \left[f(\eta_t) \right]
					= \int f(\eta) \: \pi(\mathrm{d}\eta).
				\end{equation*}
		\end{itemize}
	\item Rozważmy model epidemii na grafie $G$. Pokaż, że jeżeli
		\begin{equation*}
			1/\lambda > \max_{x} \mathrm{deg}(x),
		\end{equation*}
		to proces ten posiada tylko jedną miarę stacjonarną.
	\item Niech $N=(N(t)_{t \in \mathbb{R}_+}$ będzie jednorodnym procesem Poissona z parametrem 
		$\lambda>0$. Niech $E$ będzie niezależną od $N$ zmienną losową. Pokaż, że
		$N(t+E)-N(E)$ jest jednorodnym procesem Poissona z parametrem $\lambda>0$.

	\item Niech $\{\xi_k\}_{k \in \mathbb{N}}$ będzie ciągiem iid z rozkładem wykładniczym z 
		parametrem $\lambda>0$. Niech $S_n = \sum_{k=1}^n\xi_k$ i niech
		\begin{equation*}
			\nu(t) = \inf \{ k \: : \: S_k>t\}.
		\end{equation*}
		Pokaż, że dla każdego $t>0$ zmienna $t-S_{\nu(t)-1}$ ma ten sam rozkład co 
		$\xi_1\wedge t$.

\end{enumerate}
\end{document}
