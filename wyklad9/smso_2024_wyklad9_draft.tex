\documentclass{article}

%\usepackage{MyPackage}
\usepackage{Stochastyczne_modele_systemow_oddzialujacych_Package}
\begin{document}
SMUO 2024

wykład 9: Konstrukcja systemów spinowych

Dla $x \in V$ oraz $\eta \in \{0,1\}^V$ definiujemy $\eta^{(x)} \in \{0,1\}^V$
wzorem
\begin{equation*}
	\eta^{(x)}(y) = \left\{ 
	\begin{array}{cc} \eta(y), & y \neq x \\ 1-\eta(x), & y=x \end{array}\right..
\end{equation*}
Dla $f$ pochodzącego z odpowiedniego podzbioru  $C_0(\{0,1\}^V)$ chcemy
położyć
\begin{equation}\label{eq:4:defL}
	Lf(\eta) = \sum_{x\in V} c(x, \eta)\left[f\left(\eta^{(x)}\right) - f(\eta)\right].
\end{equation}
Okazuje się, że dokładne napisanie dziedziny jest problematyczne. Aby obejść tę trudność 
rozważmy
\begin{equation}\label{eq:4:defD}
	D = \left\{ f \in C(\{0,1\}^V) : \|f\|_o := \sup_{\eta} \sum_x \left|f\left(\eta^{(x)}\right) - f(\eta)\right| < \infty \right\}.
\end{equation}
\subsection*{Konstrukcja systemów spinowych cd.}

	Aby sprawdzić (GI4) musimy wyprowadzić ograniczenie 
	dla rozwiązań równania $f - \lambda Lf = g$. Niech
	\[
		\epsilon = \inf_{u, \eta} [c(u, \eta) + c(u, \eta_u)] \quad \text{oraz} 
		\quad \gamma(x, u) = \sup_\eta |c(x, \eta_u) - c(x, \eta)|.
	\]
	Zauważmy, że $\gamma(x,u)$ mierzy stopień, w jakim intensywność zmiany w miejscu $x$ 
	zależy od konfiguracji w miejscu $u$. 
	Niech $\ell_1(V)$ będzie przestrzenią Banacha 
	funkcji $\alpha : V \to \mathbb{R}$, które spełniają
	\[
		||\alpha|| := \sum_x |\alpha(x)| < \infty.
	\]
	Macierz $\gamma$ definiuje operator $\Gamma$ na $\ell_1(S)$ przez
	\[
		\Gamma \alpha(u) = \sum_{x: x \neq u} \alpha(x) \gamma(x, u).
	\]
	Operator ten jest dobrze zdefiniowany i ograniczony, pod warunkiem że
	\[
		M := \sup_x \sum_{u: u \neq x} \gamma(x, u) < \infty,
	\]
	a wtedy $||\Gamma|| = M$.


	Dla $f \in C(\{0,1\}^V)$ i $x \in S$, niech
	\[
		\Delta f(x) = \sup_{\eta} \left|f\left(\eta^{(x)}\right) - f(\eta)\right|.
	\]
	Wtedy $\|f\|_o = ||\Delta f||_{l_1(V)}$. Oto oszacowanie, którego potrzebujemy.

	\begin{fak}\label{fak:4.2}
		Załóżmy, że spełniony jest jeden z warunków
		\begin{itemize}
    			\item[(a)] $f \in D$,
    			\item[(b)] $f$ jest ciągła i
				\begin{equation}\label{eq:4.3}
			c(x, \cdot) \equiv 0 
			\text{ dla wszystkich oprócz skończonej liczby } x \in V.
		\end{equation}
		\end{itemize}
	Wówczas jeśli $f - \lambda Lf = g \in D$, $\lambda > 0$, 
	oraz $\lambda M < 1 + \lambda \epsilon$, to
	\begin{equation}\label{eq:4.4}
		\Delta f \leq \left[ (1 + \lambda \epsilon)I - \lambda \Gamma \right]^{-1} \Delta g,
	\end{equation}
	gdzie nierówność zachodzi współrzędna po współrzędnej, a odwrotność 
	jest zdefiniowana przez nieskończony szereg
	\begin{equation}\label{eq:4.5}
		\left[ (1 + \lambda \epsilon)I - \lambda \Gamma \right]^{-1} \alpha 
		= \frac{1}{1 + \lambda \epsilon} \sum_{k=0}^{\infty} 
		\left( \frac{\lambda}{1 + \lambda \epsilon} \right)^k \Gamma^k \alpha.
	\end{equation}
\end{fak}
\begin{proof}
	Zauważmy, że szereg w \eqref{eq:4.5} jest zbieżny dla $\alpha \in \ell_1(V)$ 
	na mocy założenia $\lambda M < 1 + \lambda \epsilon$. 
	Pisząc $f - \lambda Lf = g$ w punktach $\eta$ oraz $\eta^{(u)}$, odejmując i
	zauważając że $(\eta^{(u)})^{(u)} = \eta$, otrzymujemy
	\begin{multline}\label{eq:4.6}
		[f(\eta^{(u)}) - f(\eta)][1 + \lambda c(u, \eta) + \lambda c(u, \eta^{(u)})] 
		= [g(\eta^{(u)}) - g(\eta)]\\
		+ \lambda \sum_{x:x \neq u} \left\{ c(x, \eta^{(u)}) [f((\eta^{(u)})^{(x)}) 
		- f(\eta^{(u)})] - c(x, \eta)[f(\eta^{(x)}) - f(\eta)] \right\}.
	\end{multline}
	Ponieważ wartości $f(\eta^{(u)}) - f(\eta)$, gdy $\eta$ zmienia się a $u$ jest ustalone, 
	tworzą zbiór symetryczny, a ta różnica jest funkcją ciągłą $\eta$, 
	dla każdego $u$ istnieje takie $\eta$, że
	\[
		f(\eta^{(u)}) - f(\eta) = \sup_{\zeta} |f(\zeta^{(u)}) - f(\zeta)| = \Delta f(u).
	\]
	Stąd,
	\[
		f(\zeta^{(u)}) - f(\zeta) \leq f(\eta^{(u)}) - f(\eta)
	\]
	dla każdej $\zeta$. Stosując to dla $\zeta = \eta^{(x)}$ i przekształcając, otrzymujemy
	\[
		f((\eta^{(u)})^{(x)}) - f(\eta^{(u)}) 
		= f((\eta^{(x)})^{(u)}) - f(\eta^{(u)}) \leq f(\eta^{(x)}) - f(\eta),
	\]
	Używając tej nierówności w \eqref{eq:4.6},
	\begin{multline}\label{eq:4.7}
		\Delta f(u)(1 + \lambda \epsilon) \leq 
		\Delta f(u)[1 + \lambda c(u, \eta) + \lambda c(u, \eta^{(u)})] \\
		\leq \Delta g(u) + \lambda \sum_{x:x \neq u} 
		\left[ c(x, \eta^{(u)}) - c(x, \eta) \right] [f(\eta^{(x)}) - f(\eta)]
		\\ \leq \Delta g(u) + \lambda \sum_{x:x \neq u} \gamma(x,u) \Delta f(x).
	\end{multline}
	Jeśli \eqref{eq:4.3} zachodzi, to tylko skończona liczba wyrazów po 
	prawej stronie jest niezerowa, więc przy któregokolwiek z założeń faktu
	$\Gamma \Delta_f$ jest dobrze określona.
	Dlatego \eqref{eq:4.7} można zapisać jako
	\[
		(1 + \lambda \epsilon) \Delta f \leq \Delta g + \lambda \Gamma \Delta f.
	\]
	Iteracja tej nierówności prowadzi to do
	\[
		\Delta f \leq \frac{1}{1 + \lambda \epsilon} 
		\sum_{k=0}^{n} \left( \frac{\lambda}{1 + \lambda \epsilon} \right)^k 
		\Gamma^k \Delta g + \left( \frac{\lambda}{1 + \lambda \epsilon} \right)^{n+1} 
		\Gamma^{n+1} \Delta f.
	\]
	Jeżeli rozważymy teraz $n \to \infty$, dostaniemy \eqref{eq:4.4}.
\end{proof}


\begin{thm} \label{thm:4.3}
	Załóżmy, że $M < \infty$. 
	Wtedy $\overline{L}$ jest generatorem infinitezymalnym półgrupy 
	Fellera $T=(T(t))_{t \in \mathbb{R}_+}$. 
	Ponadto,
	\begin{equation}\label{eq:4.8}
		\Delta T(t)f \leq e^{-t \epsilon} e^{t \Gamma} \Delta f.
	\end{equation}
	W szczególności, jeśli $f \in D$, to $T_tf \in D$ oraz
	\begin{equation}\label{eq:4.9}
		\|T(t)f\|_o \leq e^{(M - \epsilon)t} \|f\|_o.
	\end{equation}
\end{thm}

\begin{proof}
	Własności (GI1), (GI2), (GI3) i (GI5) z Definicji~\ref{defn:3:12} 
	zachodzą dla $(L, D)$ są  i są dziedziczone przez $\overline{L}$ z Faktu~\ref{fak:3:30}. 
	Aby sprawdzić warunek (GI4) weźmy  wstępujący ciąg $V_n\subseteq V$ taki, że
	$\bigcup_nV_n=V$. Niech
	\begin{equation}\label{eq:4.10}
		L_n f(\eta) = 
		\sum_{x \in V_n} c(x, \eta) \left[f\left(\eta^{(x)}\right) - f(\eta)\right], 
		\quad f \in C\left(\{0,1\}^V\right).
	\end{equation}
	To jest generator dla systemu spinowego, w którym współrzędne
	\[
		(\eta(x) : x \notin V_n)
	\]
	są stałe w czasie. 
	Ponieważ $L_n$ jest ograniczonym generatorem, spełnia
	\[
		\mathcal{R}(I - \lambda L_n) = C(\{0,1\}^V)
	\]
	dla dostatecznie małych $\lambda > 0$. 
	Dla $g \in D$, możemy zdefiniować $f_n \in C(\{0,1\}^V)$ 
	przez $f_n - \lambda L_n f_n = g$. 
	Ponieważ $L_n$ spełnia \eqref{eq:4.3}, jeśli $\lambda$ jest wystarczająco małe, 
	tak że $\lambda M < 1 + \lambda \epsilon$, wtedy $f_n \in D$ zgodnie z Faktem~\ref{fak:4.2}. 
	W związku z tym możemy położyć
	\[
		g_n = f_n - \lambda L f_n \in \mathcal{R}(I - \lambda L).
	\]
	Niech $K = \sup_{x, \eta} c(x, \eta) <\infty$, wtedy z Faktu~\ref{fak:4.2},
	\begin{multline}\label{eq:4.11}
		\|g_n - g\| 
		= \lambda ||(L - L_n) f_n|| 
		\leq \lambda K \sum_{x \notin V_n} \Delta f_n(x)\\
		\leq \lambda K \sum_{x \notin V_n} 
		\left[ (1 + \lambda \epsilon)I - \lambda \Gamma \right]^{-1} \Delta g(x).
	\end{multline}
	Ponieważ $\Delta g \in \ell_1(V)$, prawa strona \eqref{eq:4.11} 
	dąży do zera, gdy $n \to \infty$, więc $g_n \to g$. 
	Stąd $g \in \mathrm{cl}(\mathcal{R}(I - \lambda L))$, więc wnioskujemy, 
	że $D \subseteq \mathrm{cl}(\mathcal{R}(I - \lambda L))$. 
	Ponieważ $D$ jest gęste w $C(\{0,1\}^V)$, 
	widzimy, że $\mathcal{R}(I - \lambda L)$ jest również gęste. 
	Zgodnie z Faktem~\ref{fak:3:30}, 
	$\mathcal{R}(I - \lambda \overline{L})$ musi być domkniętym podzbiorem  $C(\{0,1\}^V)$.
	Zatem
	\[
		\mathcal{R}(I - \lambda \overline{L}) = C(\{0,1\}^V)
	\]
	To kończy weryfikację, że $\overline{L}$ jest generatorem infinitezymalnym.


	Przechodząc do drugiego stwierdzenia, zapiszmy \eqref{eq:4.4} jako
	\[
		\Delta_{(I - \lambda L)^{-1}} g 
		\leq \left[ (1 + \lambda \epsilon)I - \lambda \Gamma \right]^{-1} \Delta g,
	\]
	a następnie iterujmy, aby uzyskać
	\[
		\Delta_{(I - \frac{t}{n} L)^{-1}} g \leq 
		\left[ \left( 1 + \frac{t}{n} \epsilon \right) I - \frac{t}{n} \Gamma \right]^{-n} 
		\Delta g.
	\]
	Przechodząc do granicy otrzymujemy \eqref{eq:4.8}.
\end{proof}



\end{document}
