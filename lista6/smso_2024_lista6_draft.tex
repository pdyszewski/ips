\documentclass{article}

\usepackage{MyPackage}

\begin{document}
SMUO 2024

lista 6: Kilka zaginionych faktów

\begin{enumerate}
	\item Niech $T = (T_{t})_{t \in \mathbb{R}_+}$ będzie półgrupą Fellera na $S$.
		Pokaż, że dla każdego $x \in S$ i każdego $t \geq 0$ istnieje miara $\mu_{t,x}$
		na $S$ taka, że
		\begin{equation*}
			T_tg(x) = \int g(y) \: \mu_{t,x}(\mathrm{d} y).
		\end{equation*}
		\textsc{Wskazówka}: Rozważ jednopunktowe uzwarcenie $S$.
	\item Wywnioskuj z poprzedniego zadania, że $\| T_t\|\leq 1$.
	\item Niech $T = (T_t)_{t \in \mathbb{R}_+}$ będzie półgrupą Fellera.
		Pokaż, że dla każdego $t \geq 0$,
		\begin{equation*}
			\inf_{x \in S} T_tf(x) \geq  
			\inf_{x \in S} f(x) .
		\end{equation*}
	\item Niech $(\mathbf{P}, \mathbb{F})$ będzie procesem Fellera. 
		Rozważmy ograniczoną funkcję mierzalną
		$\varphi \colon \mathbb{R}_+ \times \Omega \to \mathbb{R}$.
		Pokaż, że
		dla $\mathbb{F}$-czasu zatrzymania $\tau$ zachodzi
		\begin{equation*}
			\mathbf{E}_x \left[ \varphi(\tau, \theta_{\tau})  | \mathcal{F}_\tau\right] 
			= \Phi(\tau, X_{\tau}) \quad \mathbf{P}_x-p.w.
		\end{equation*}
		gdzie
		\begin{equation*}
			\Phi(t,x) = \int_\Omega \varphi(t, \omega) \mathbf{P}_x(\mathrm{d}\omega).
		\end{equation*}
	\item Pokaż, że jeżeli $A \colon C_0(S) \to C_0(S)$ jest ograniczonym operatorem,
		to $\mathcal{R}(I-\epsilon A) = C_0(S)$ dla dostatecznie małych $\epsilon>0$.
	\item Niech $(L, \mathcal{D}(L))$ będzie generatorem infinitezymalnym. Dla $\epsilon>0$ rozważmy
		\begin{equation*}
			L_\epsilon = L(I-\epsilon L)^{-1}, \qquad 
			T_\epsilon(t) = \sum_{n=0}^\infty \frac{t^nL_\epsilon^n}{n!}.
		\end{equation*}
		Pokaż, że $T_\epsilon$ jest półgrupą Fellera z generatorem $L_\epsilon$.
\end{enumerate}
\end{document}
