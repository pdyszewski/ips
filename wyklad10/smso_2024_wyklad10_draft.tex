\documentclass{article}

%\usepackage{MyPackage}
\usepackage{Stochastyczne_modele_systemow_oddzialujacych_Package}
\begin{document}
SMUO 2024

wykład 10: Procesy dualne

\section*{Procesy dualne}

	Często zdarza się, że wartość oczekiwaną związaną z jednym skomplikowany procesem
	jesteśmy w stanie wyrazić jako wartość oczekiwaną związaną z innym procesem, który jest 
	o wiele poostrzy.
	Tego typu relacje pozwalają na przydatne reprezentacje 
	wielkości występujących w procesach Fellera. 

	\begin{defn}\label{defn:3-41}
		Niech $X_1=(X_1(t))_{t \in \mathbb{R}}$ i $X_2=(X_2(t))_{t \in \mathbb{R}}$ 
		będą procesami Fellera odpowiednio na przestrzeniach $S_1$ i $S_2$. 
		Dla mierzalnej i ograniczonej funkcji $H$ na $S_1 \times S_2$, 
		procesy te są nazywane dualnymi względem $H$, jeśli
		\begin{equation}\label{eq:3-30}
			\Ebf_{x_1} [H(X_1(t), x_2)] = \Ebf_{x_2} [H(x_1, X_2(t))]
		\end{equation}
		dla każdego $t \geq 0$ oraz $x_i \in S_i$.
	\end{defn}

	Powyższe pojęcie w zupełnej ogólności jest problematyczne. 
	Naturalnym jest oczekiwać, że~\eqref{eq:3-30} mówi coś o relacji między
	$X_1$ oraz $X_2$. 
	Zauważmy, że każde dwa procesy są dualne względem funkcji stałej. 
	Jednakże charakter relacji między $X_1$ a $X_2$ zależy bardzo mocno
	od konteksty i podyktowany jest prze funkcję $H$.

	\begin{pd}
		Niech $X_a$ oraz $X_r$ będą ruchami Browna na $S_1=S_2=[0, +\infty)$ odpowiednio
		zabitymi i odbitymi w zerze. Procesy te są dualne względem funkcji
		\begin{equation*}
			H(x,y) = \mathbf{1}_{\{ x \leq y \}}.
		\end{equation*}
		Wówczas relacja~\eqref{eq:3-30} zapisuje się jako
		\begin{equation*}
			\Pbf_{x_1}[X_1(t) \leq x_2] = \Pbf_{x_2}[x_1 \leq X_2(t)].
		\end{equation*}
		W przypadku wspomnianych wersji ruchu Browna
		\begin{equation*}
			\Pbf_{x}[X_a(t) \leq y] = \Pbf_{y}[x \leq X_r(t)].
		\end{equation*}
		Oba prawdopodobieństwa są równe
		\begin{equation*}
			\mathbb{P}[B_t\geq x-y]+ \mathbb{P}[B_t\geq x+y],
		\end{equation*}
		gdzie $B=(B_t)_{t \in \mathbb{R}_+}$ jest standardowym ruchem Browna na $\mathbb{R}$.
		Dokładne sprawdzenie wspomnianej równości pozostawiamy jako zadanie.
	\end{pd}

	Załóżmy teraz, że $H$ jest ciągła.
	W świetle definicji procesy Fellera $X_1$ oraz $X_2$ z półgrupami Fellera odpowiednio 
	$T_1=(T_1(t))_{t \in \mathbb{R}_+}$ oraz $T_2=(T_2(t))_{t \in \mathbb{R}_+}$
	są $H$-dualne wtedy i tylko wtedy, gdy
	\begin{equation*}
		T_1(t)H(\cdot, s_2)(s_1) = T_2(t)H(s_1, \cdot) (s_2)
	\end{equation*}
	dla wszystkich $t \geq 0$, $s_1\in S_1$ oraz $s_2 \in S_2$. W przypadku,
	gdy definiujemy procesy Fellera przez ich opis infinitezymalny wygodniejsze jest kryterium
	wyrażone w terminach generatorów.

	\begin{thm}\label{thm:3.42} 
		Niech $X_1$ i $X_2$ będą generowane odpowiednio przez $(L_1, \mathcal{D}(L_1))$ 
		oraz $(L_2, \mathcal{D}(L_2)$,
		Załóżmy, że dla każdych $s_1\in S_1$ oraz $s_2 \in S_2$
		\begin{equation*}
			H(\cdot, s_2) \in \mathcal{D}(L_1) \quad \mbox{oraz} 
			\quad H(s_1, \cdot) \in \mathcal{D}(L_2).
		\end{equation*}
		Jeżeli dodatkowo
		\begin{equation*}
			L_1 H(\cdot, s_2)(s_1) = L_2 H(s_1, \cdot)(s_2)
		\end{equation*}
		dla wszystkich $x_1 \in S_1$ oraz $x_2 \in S_2$. 
		Wówczas $X_1$ i $X_2$ są dualne względem $H$.
	\end{thm}

	\begin{proof}
		Rozumowanie przeprowadzimy jedynie w przypadku przeliczalnej $S_2$ i ograniczonego 
		$L_2$. Wówczas $X_2$ jest łańcuchem Markowa z $q$-macierzą $q=(q(x,y))_{x,y\in S_2}$.
		Przypomnijmy, że wówczas
		\begin{equation*}
			T_2(t)f(s_2)=\Ebf_{s_2}[f(X_2(t)] = \sum_{y \in S_2} \Pbf_{s_2}[X_2(t) =y] f(y).
		\end{equation*}
		Generator $L_2$ zadany jest wówczas przez
		\begin{equation*}
			L_2 f(s_2) = \left.\frac{\mathrm{d}}{\mathrm{d}t} \right|_{t=0} T_2(t)f(s_2)
			= \sum_{y \in S_2}\left.\frac{\mathrm{d}}{\mathrm{d}t} \right|_{t=0} 
				\Pbf_{s_2}[X_2(t) =y] f(y)
			=\sum_{y \in S_2} q(s_2, y)f(y).
		\end{equation*}
		Rozważmy
		\begin{equation}\label{eq:3-31}
			u(t, x_1, x_2) = \mathbf{E}_{x_1}H(X_1(t), x_2) = T_1(t) H(\cdot, x_2)(x_1)
		\end{equation}
		Na mocy Twierdzenia~\ref{thm:3:16},
		\begin{multline}
			\frac{\mathrm{d}}{\mathrm{d}t} u(t, x_1, x_2) 
			= T_1(t)L_1 H(\cdot, x_2)(x_1) = T_1(t)L_2 H(x_1, \cdot)(x_2) \\
			= \sum_{y\in S_2} q(x_2, y) T_1(t) H(\cdot, y)(x_1) 
			= \sum_y q(x_2, y) u(t, x_1, y)
			= L_2 u(t, x_1, \cdot)(x_2).
		\end{multline}
		Dodatkowo $u(0, x_1, x_2) = H(x_1, x_2)$.
		Z drugiej strony funkcja 
		\begin{equation*}
			v(t,x_1, x_2) = T_2(t)H(x_1, \cdot)(x_2)
		\end{equation*}
		również spełnia $v(0,x_1, x_2) = H(x_1, x_2)$. Wystarczy zatem uzasadnić 
		jedyność tego zagadnienia. Rozważmy w tym celu $h = v-u$. Wówczas
		\begin{equation*}
			h(t,x_1, x_2) = \int_0^t L_2 h(s, x_1, \cdot ) \mathrm{d}s
		\end{equation*}
		Niech 
		\begin{equation*}
			h^*(t,x_1) = \sup_{ s \leq t, x_2 \in S_2} h(s,x_1, x_2).
		\end{equation*}
		Wówczas
		\begin{equation*}
			h^*(t,x_1) \leq \int_0^t \|L_2\| h^*(s, x_1) \mathrm{d}s.
		\end{equation*}
		Powyższa nierówność zwija się do
		\begin{equation*}
			\frac{\mathrm{d}}{\mathrm{d}t} 
			\left(e^{- \|L_2\|t} \int_0^t h^*(s, x_1) \mathrm{d}s \right)\leq 0.
		\end{equation*}
		Co po całkowaniu daje
		\begin{equation*}
			e^{- \|L_2\|t} \int_0^t h^*(s, x_1) \mathrm{d}s\leq 0.
		\end{equation*}
		Skoro lewa strona jest niewątpliwie nieujemna, to $h \equiv 0$ za co za tym idzie 
		$u\equiv v$. Ostatnie równość jest równoważna z dowodzoną tezą.
\end{proof}

\begin{pd}
	Niech $X_1$ i $X_2$ będą spacerami losowymi na $\mathbb{Z}$ z $q$-macierzami odpowiednio
	\begin{equation*}
		q_1(x, x+1) = \beta, \quad q_1(x, x-1) = \delta
	\end{equation*}
	oraz
	\begin{equation*}
		q_2(x, x+1) = \delta, \quad q_1(x, x-1) = \beta
	\end{equation*}
\end{pd}

\section{The Voter model}

Niech $V$ będzie przeliczalnym zbiorem z topologia dyskretną. Chcemy modelować
proces rozwoju opinii wśród osobników reprezentowanych przez elementy $V$. 
Zakładać będziemy, że w każdej chwili czasu $ t \geq 0$ każdy osobnik $x \in V$ 
reprezentuje jedną 
z dwóch opinii $\eta_t(x) \in \{0,1\}$ na zadany temat. 
Załóżmy, że dane są nieujemne liczby $q(x, y)$ dla $x \neq y$. 
Wielkość $q(x,y)$ będzie intensywnością z jaką $x$ przejmuje opinię $y$ o ile oba osobniki 
reprezentują różne opinie. Zakładać będziemy, że
\[
	M = \sup_{x\in V} \sum_{u : u \neq x} q(x, u) < \infty.
\]
Model głosowania (the Voter model) $\eta_t$ to system spinowy z
\[
	c(x, \eta) = \sum_{y : \eta(y) \neq \eta(x)} q(x, y).
\]
Innymi słowy jest to proces Fellera generowany przez
\begin{equation*}
	Lf(\eta) = \sum_{x \in V} c(x, \eta) \left( f\left(\eta^{(x)} \right) - f(\eta) \right).
\end{equation*}
Techniczne szczegóły związane z dziedziną $L$ zostały przedyskutowane w poprzednim rozdziale.  
Najważniejsze jest, że z Twierdzenia~\ref{thm:4.3} wiemy, że proces ten jest dobrze określony 
(domknięcie $L$ jest generatorem infinitezymalnym). 

\begin{pd}
	Załóżmy, że $V$ jest wyposażone w strukturę grafu o ograniczonym stopniu. 
	Chcemy modelować przypadek w którym każdy z wierzchołków $x \in V$ może wchodzić
	w interakcję jedynie ze swoimi bezpośrednimi sąsiadami (i to od nich zapożycza opinie).
	Rozważmy $q(x,y) = \mathbf{1}_{x \sim y}$.
	Wówczas
	\begin{equation*}
		M = \sup_{x \in V} \sum_{u: u \neq x} q(x,u) = \sup_{x\in V} \mathrm{deg}(x) <\infty.
	\end{equation*}
\end{pd}

Zauważmy, że model głosowania posiada dwa stany stacjonarne $\eta \equiv 1$ oraz $\eta \equiv 0$. 
Naszym głównym celem jest sprawdzenie, czy istnieją inne (nietrywialne) rozkłady stacjonarne.


Aby tego dokonać posłużymy się procesem dualnym do $(\eta_t)_t$.
Ustalmy $t \geq 0$ i $x \in V$. Skoro przy zmianach opinia w $x$ jest zapożyczana od innych 
osobników, chcąc zbadać wartość $\eta_t(x)$ rozważmy $t_1$-moment ostatniej zmiany opinii
przez $x$, czyli
\begin{equation*}
	t_1 = \sup_{s \leq t} \{ \eta_{s-}(x) \neq \eta_s(x) \}
\end{equation*}
Jeżeli zbiór czasów pod kresem górnym jest pusty, to $x$ nie zmienił zdania na odcinku
czasu $[0,t]$, więc $\eta_t(x) = \eta_0(x)$.
W chwili $t_1$, $x$ przyjął tę samą opinię co pewien $x_1$ (co się dzieje z intensywnością $q(x, x_1)$),
czyli $\eta_t(x) = \eta_{t_1}(x_1)$.
Chcąc ustalić wartość $\eta_{t_1}(x_1)$ rozważamy ostatni moment, w którym $x_1$ zmienił opinię
\begin{equation*}
	t_2 = \sup_{s \leq t} \{ \eta_{s-}(x_1) \neq \eta_s(x_1) \}
\end{equation*}
Jeżeli zbiór pod kresem górnym jest pusty, to $x_1$ na przedziale czasowym $[0,t_1]$ nie zmienił
zdania i $\eta_t(x) = \eta_{t-1}(x_1) = \eta_0(x_1)$. Postępując iteracyjne dostajemy ciąg czasów
$t\geq t_1 > t_2>\ldots >t_N$ taki, że
\begin{equation*}
	\eta_t(x) =\eta_{t_1}(x_1) = \ldots = \eta_{t_N}(x_N) = \eta_0(x_N).
\end{equation*}
Przy czym przejście z $x_k$ do $x_{k+1}$ dzieje się z intensywnością $q(x_k, x_{k+1}$.
Skonstruowana w ten sposób ścieżka $(x, x_1, \ldots, x_N)$ ma taki sam rozkład 
jak ścieżka łańcucha Markowa $Y_x = (Y_x(t))_{t \in \mathbb{R}_+}$ z $q$-macierzą 
$(q(x,y))_{x,y \in V}$. Oznacza to, że
\begin{equation*}
	\eta_t(x) \stackrel{d}{=} \eta_0(Y_x(t)).
\end{equation*}
Chcąc zbadać teraz rozkład łączny $(\eta_t(x), \eta_t(y)$ dla $x,y \in V$ możemy wykonać 
podobną konstrukcję na podstawie zmian opinii któregokolwiek z elementów pary. Dostaniemy w ten sposób 
ciąg czasów $t \geq s_1> s_2 > \ldots >s_M\geq 0$ taki, że 
\begin{equation*}
	\eta_t(x) =\eta_{s_1}(x_1) = \ldots = \eta_{s_M}(x_M) = \eta_0(x_M).	
\end{equation*}
oraz
\begin{equation*}
	\eta_t(y) =\eta_{s_1}(y_1) = \ldots = \eta_{s_M}(y_M) = \eta_0(y_M).	
\end{equation*}
Kluczowa jest następująca własność. Jeżeli $x_j=y_j$ dla pewnego $j\geq 1$, to $x_k=y_k$ dla 
wszystkich $k \geq j$. Istotnie, jeżeli $j$ jest najmniejszą taką liczbą, że
$x_j=y_j$, to oznacza to że $x_{j-1}$ przejął opinię $x_j$. Po czym zanim $x_j$ zmienił opinię, to
$y_{j-1}$ przejął opinię $x_j$.
Oznacza to, że
\begin{equation*}
	(\eta_t(x), \eta_y(y)) \stackrel{d}{=} (\eta_0(Y_x(t)), \eta_0(Y_y(t))),
\end{equation*}
gdzie $Y_x$ i $Y_y$ są łańcuchami Markowa na $V$ z zadaną $q$-macierzą takimi, że
jeżeli w pewnym momencie się spotkają, to od tego momentu zaczynają się poruszać się razem. 
Podobny komentarz możemy napisać dla wektora $\eta_x(t)$ dowolnej długości: jego rozkład będziemy
mogli wyrazić przez kolekcję spacerów losowych na $V$, które się zlewają w momencie spotkania.
Aby wprowadzić ten proces bardziej formalnie, rozważmy $S_2(N)$ - zbiór wszystkich
$A \subseteq V$ o liczebności nie większej niż $N$. Rozważmy $\{Q(A,B)\}_{A,B \in S_2}$ dane przez
\[
Q(A, (A \setminus \{x\}) \cup \{y\}) = q(x, y), \quad x \in A, \, y \notin A;
\]
oraz
\[
Q(A, A \setminus \{x\}) = \sum_{y \in A, y \neq x} q(x, y), \quad x \in A.
\]
Taki wybór instancyjności przejść odpowiada dokładnie zlewającym się spacerom losowym.
Generator  takiego procesu jest ograniczony z normą
\[
	\sum_{B \neq A} Q(A, B) = \sum_{x \in A} \sum_{y \neq x} q(x, y) \leq M N.
\]
Pokażemy, że zlewające się spacery losowe $A=(A_t)_{t \in\mathbb{R}}$ 
(proces o intensywnościach danych powyżej)
jest dualny do modelu głosowania $(\eta_t)_{t \in \mathbb{R}}$ z funkcją
\[
H(\eta, A) = \prod_{x \in A} \eta(x) = 1_{\{\eta = 1 \text{ on } A\}},
\]
Trajektorie $|A_t|$ są nierosnące, co czyni go bardzo użytecznym w badaniu modelu głosowania, 

\begin{thm}\label{thm:4.32}
	Procesy $(\eta_t)_t$ i $(A_t)_t$ są dualne względem $H(\eta, A)$.
\end{thm}
\begin{proof}
	Sprawdzimy, że zachodzą założenia Twierdzenia~\ref{thm:3.42}.
	Niech $L$ będzie generatorem modelu głosowania. Ponieważ $H(\eta, A)$ zależy od $\eta$ 
	tylko poprzez $\{\eta(x), x \in A\}$,
	\begin{multline}
		L H(\cdot, A)(\eta) 
		= \sum_{x \in A, y \in S \atop \eta(y) \neq \eta(x)} 
			q(x, y) [H(\eta_x, A) - H(\eta, A)]\\
		= \sum_{x \in A, y \in S \atop \eta(y) \neq \eta(x)} 
			q(x, y) [1 - 2\eta(x)] H(\eta, A \setminus \{x\})\\
		= \sum_{x \in A, y \in S} 
			q(x, y) H(\eta, A \setminus \{x\}) [\eta(y) - \eta(x)]\\
		= \sum_{x \in A, y \in S} 
			q(x, y) [H(\eta, (A \setminus \{x\}) \cup \{y\}) - H(\eta, A)]\\
		= \sum_{B} q(A, B) [H(\eta, B) - H(\eta, A)].
	\end{multline}
\end{proof}
\end{document}
