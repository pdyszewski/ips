\documentclass{article}

%\usepackage{MyPackage}
\usepackage{Stochastyczne_modele_systemow_oddzialujacych_Package}
\begin{document}
SMUO 2024

wykład 7: Rozkłady stacjonarne, zaburzenia ruchu Browna

\section{Rozkłady stacjonarne}

	Interesować nas będzie asymptotyczne zachowanie procesów Fellera. 
	Podobnie jak w przypadku łańcuchów Markowa w czasie dyskretnym
	rozkłady graniczne są niezmiennicze ze względu na funkcje przejścia.
	Przez $\Sigma$ oznaczać będziemy $\sigma$-ciało zbiorów borelowskich $S$,
	czyli najmniejsze $\sigma$-ciało zawierające wszystkie otwarte podzbiory $S$. 
	Skoro $S$ jest ośrodkowa, to $\Sigma$ jest generowane przez wszystkie kule otwarte.

	Dla Procesu Fellera $(\mathbf{P}, \mathbb{F})$ oraz rozkładu prawdopodobieństwa 
	$\mu$ na $S$ definiujemy miarę probabilistyczną $\mathbf{P}_\mu$ na $(S, \Sigma)$ 
	wzorem
	\begin{equation*}
		\mathbf{P}_\mu[A] 
		= \int_S \mathbf{P}_x[A] \: \mu(\mathrm{d}x), 
		\qquad A \in \mathcal{F}.
	\end{equation*}
	W tym miejscu zachęcamy czytelnika do wprawdzenia, że odwzorowanie $x \mapsto \mathbf{P}_x[A]$ 
	jest mierzalne dla $A \in \mathcal{F}$. Miara $\mathbf{P}_\mu$ to rozkład procesu Markowa
	przy rozkładzie początkowym $\mu$. 
	
	\begin{defn}
		Niech $(\mathbf{P}, \mathbb{F})$ będzie procesem Fellera
		Rozkład prawdopodobieństwa $\pi$ na $(S, \Sigma)$ nazywamy rozkładem stacjonarnym 
		jeżeli
		\begin{equation*}
			\mathbf{P}_\pi \left[ X(t) \in A \right] = \pi(A)
		\end{equation*}
		dla każdego $A \in \Sigma$.
	\end{defn}

	Chcielibyśmy wiedzieć, jak określić na podstawie generatora, 
	czy miara prawdopodobieństwa na $S$ jest stacjonarna dla procesu Fellera. 
	Z tego powodu przepiszemy powyższą definicję w terminach półgrupy.
	Jeśli $\mu$ jest miarą prawdopodobieństwa na $S$, rozkład procesu w czasie $t$, 
	gdy rozkład początkowy jest $\mu$, 
	oznaczamy przez $\mu T(t)$. Spełnia on zależność
	\[
		\int f \, \mathrm{d}(\mu T(t)) = \int T(t)f \, \mathrm{d} \mu = \Ebf_\mu \left[ f(X(t)) \right] 
	\]
	dla $f \in C_0(S)$. Tutaj $\mathbf{E}_\mu$ to wartość oczekiwana odpowiadająca $\mathbf{P}_\mu$. 
	Równoważnie
	\begin{equation*}
		\mathbf{E}_\mu[Y] = \int_X \Ebf_x[Y] \: \mu(\mathrm{d}x)
	\end{equation*}
	dla każdej ograniczonej zmiennej losowej $Y \colon \Omega \to \mathbb{R}$.

	\begin{defn}
		Miara prawdopodobieństwa $\mu$ na $S$ jest stacjonarna dla procesu Fellera z półgrupą $T(t)$, 
		jeśli $\mu T(t) = \mu$ dla wszystkich $t \geq 0$, tzn. jeśli
		\begin{equation}\label{eq:3:28}
			\int T(t)f \, \ud\mu 
			= \int f \, \ud\mu 
			\quad \text{dla wszystkich} 
			\ f \in C_0(S) \ \text{i} \ t \geq 0.
		\end{equation}
	\end{defn}

	Będziemy używać $\mathcal{I}$ do oznaczania klasy rozkładów stacjonarnych dla procesu Fellera. 
	Z \eqref{eq:3:28} wynika, że $\mathcal{I}$ jest wypukła. 
	Zbiór punktów ekstremalnych $\mathcal{I}$ będzie oznaczany przez $\mathcal{I}_e$.

	\begin{zad}
		Pokaż, że jeśli $\mu$ jest miarą prawdopodobieństwa na $S$ 
		i $\mu T(t) \Rightarrow \nu$, to $\nu$ jest stacjonarna. 
	\end{zad}

	\begin{thm}
		Miara prawdopodobieństwa $\mu$ na $S$ jest stacjonarna 
		dla odpowiadającego procesu wtedy i tylko wtedy, gdy
		\[
			\int Lf \, \ud\mu = 0 \quad \text{dla wszystkich} \ f \in D.
		\]
	\end{thm}

	\begin{proof}
		Przypuśćmy, że $\mu$ jest stacjonarna, i weźmy $f \in \mathcal{D}(L)$. Wtedy
		\[
			\int Lf \, \ud\mu 
			= \int \lim_{t \to 0} \frac{T(t)f - f}{t} \, \ud\mu 
			= \lim_{t \to 0}  \frac{\int T(t)f \ud\mu - \int f \, \ud\mu}{t} = 0.
		\]
		Przeciwnie, przypuśćmy, że 
		$\int Lf \, \ud\mu = 0$ dla wszystkich $f \in \mathcal{D}(L)$ 
		i jeśli $f \in \mathcal{D}(L)$ oraz $f - \lambda Lf = g$, to $\int f \, d\mu = \int g \, d\mu$. 
		Iterując to, otrzymujemy
		\[
			\int (I - \lambda L)^{-n}g \, \ud\mu = \int g \, \ud\mu.
		\]
		Biorąc $\lambda = t/n$ i przechodząc z $n \to \infty$ wnioskujemy, że 
		\[
			\int T(t)g \, \ud\mu = \int g \, \ud\mu.
		\]
	\end{proof}

	Oto wystarczający warunek na istnienie rozkładu stacjonarnego.

	\begin{thm}
		Jeśli $S$ jest przestrzenią zwartą, to $\mathcal{I} \neq \emptyset$.
	\end{thm}

	\begin{proof}
		Rozważmy proces Fellera z dowolnym rozkładem początkowym $\mu$.
		Niech $\nu_n$ będzie rozkładem zmiennej 
		Zdefiniujmy miarę $\nu_n$ na $S$ poprzez warunek
		\[
			\int_S f(y) \nu_n( \mathrm{d}y) = 
			\mathbb{E} \left[ \mathbf{E}_\mu \left[ f(X_{nU}) \right] \right]=
			\mathbb{E} \left[ \int_S T_{nU}f(y) \: \mu(\mathrm{d}y) \right].
		\]
		Dla $f \in C_0(S)$, własność półgrupy daje
		\[
			\int T(t)f(y) \, \nu_n (\mathrm{d}y)=
			\mathbb{E} \left[ \int_S T_{nU+t}f(y) \: \mu(\mathrm{d}y) \right].
		\]
		tak że
		\begin{multline}\label{eq:3:29}
			\int f \, \ud\nu_n - \int T(t)f \, \ud\nu_n 
			= \int f \, \ud\nu_n - \int f \, \ud(\nu_n T(t))\\
			= \frac{1}{n} \left[ \int_0^t \int_S T(r)f \, \ud\mu \, \ud r - 
			\int_n^{n+t} \int_S T(r)f \, \ud\mu \, \ud r \right].
		\end{multline}
		Prawa strona \eqref{eq:3:29} dąży do zera gdy $n \to \infty$.

		Teraz, ponieważ $S$ jest zwarty, twierdzenie Prochorowa, implikuje, 
		że istnieje podciąg $\nu_{n_k}$ taki, że
		\[
			\nu_{n_k} \Rightarrow \nu
		\]
		dla pewnej miary prawdopodobieństwa $\nu$ na $S$. 
		Zatem, ponieważ $T(t)f \in C(S)$, możemy przejść do granicy w \eqref{eq:3:29} 
		wzdłuż ciągu $\nu_{n_k}$, aby otrzymać
		\[
			\int f \, \ud\nu = \int T(t)f \, \ud\nu.
		\]
		Ponieważ to zachodzi dla wszystkich $f \in C_0(S)$, wynika stąd, że $\nu T(t) = \nu$.
	\end{proof}

	\section{Zaburzenia ruchu Browna}

	\begin{pd}
		Rozważmy ruch Browna na $[0, \infty)$ z absorpcją w 0. 
		Niech $\tau$ będzie czasem pierwszego uderzenia w 0. Zdefiniujmy
		\[
			X_a(t) = 
\begin{cases}
X(t) & \text{jeśli } t < \tau, \\
0 & \text{jeśli } t \ge \tau,
\end{cases}
\]
oraz oznaczmy przez $L_a$ i $T_a(t)$ odpowiednio generator i półgrupę. 
Dla $f \in C_0[0, \infty)$, niech $f_o$ będzie „nieparzystym” przedłużeniem $f$ na $\mathbb{R}$:
\[
f_o(x) = 
\begin{cases}
f(x) & \text{jeśli } x \ge 0, \\
2f(0) - f(-x) & \text{jeśli } x < 0.
\end{cases}
\]
Z zasady odbicia dla każdej $g \in C_0[0,\infty)$,
\[
	\mathbf{E}_x \left[g(X(t)) \mathbf{1}_{\{ t \ge \tau \}} \right] = 
	\mathbf{E}_x \left[g(-X(t)) \mathbf{1}_{\{ t \ge \tau \}} \right].
\]
Biorąc $g=f_o$,
\[
	\mathbf{E}_x \left[f_o(X(t)) \mathbf{1}_{\{ t \ge \tau \}} \right] 
	= \mathbf{E}_x \left[f_o(-X(t)) \mathbf{1}_{\{ t \ge \tau \}} \right].
\]
Obie te wielkości są równe
\[
	\frac{1}{2} \mathbf{E}_x \left[(f_o(X(t)) + f_o(-X(t)) ) \mathbf{1}_{\{ t \ge \tau \}} \right].
\]
Ostatnie wyrażenie, z definicji $f_o$ jest równe 
\[
f(0)\mathbf{P}_x(t \ge \tau).
\]
Podsumowując dla $x \ge 0$,
\[
	T_a(t)f(x) = \mathbf{E}_x \left[f(X(t)) \mathbf{1}_{\{ t < \tau \}} \right] + 
	f(0)\mathbf{P}_x(t \ge \tau) = \mathbf{E}_x f_o(X(t)).
\]
Oczywiście $f_o \notin C(\mathbb{R})$ o ile $f(0) = 0$. Niemniej jednak, skoro
\[
f_o''(x) = 
\begin{cases}
f''(x) & \text{jeśli } x > 0, \\
-f''(-x) & \text{jeśli } x < 0,
\end{cases}
\]
wtedy $f''(0) = 0$ jest potrzebne, aby $f_o''$ było ciągłe. Wynika z tego, że
\[
\mathcal{D}(L_a) = \{f \in C_0[0, \infty) :  f'' \in C[0, \infty), f''(0) = 0\},
\]
a dla $f \in \mathcal{D}(L_a)$, $L_a f = \frac{1}{2} f''$.
\end{pd}

\begin{pd}
Rozważmy ruch Browna na $[0, \infty)$ z odbiciem w $0$. 
Proces ten jest zdefiniowany jako
\[
X_r(t) = |X(t)|,
\]
a jego generator i półgrupa będą oznaczane odpowiednio przez 
$L_r$ i $T_r(t)$. Jeśli $f \in C_0[0, \infty)$, 
niech $f_e$ będzie parzystym przedłużeniem $f$ na $\mathbb{R}$:
\[
f_e(x) = 
\begin{cases}
f(x) & \text{jeśli } x \ge 0, \\
f(-x) & \text{jeśli } x < 0.
\end{cases}
\]
Wtedy
\[
T_r(t)f(x) = \mathbf{E}_x \left[f(|X(t)|)\right] = \mathbf{E}_x f_e(X(t)) \quad \text{dla } x \ge 0.
\]
Zatem,
\[
f \in \mathcal{D}( L_r) \iff f_e \in \mathcal{D}(L).
\]
Wynika z tego, że
\[
\mathcal{D}(L_r) = \{f \in C[0, \infty) : f', f'' \in C[0, \infty), f'(0) = 0\},
\]
a dla $f \in \mathcal{D}(L_r)$, $L_r f = \frac{1}{2} f''$.
\end{pd}

\begin{pd}
Zaprezentujemy teraz ruch Browna na $[0, \infty)$ z lepkim $0$. 
Dla $c > 0$, rozważmy operator $L_c$ zdefiniowany jako $L_c f = \frac{1}{2} f''$ na
\[
\mathcal{D}(L_c) = \{f \in C_0[0, \infty) : f'' \in C[0, \infty), f'(0) = c f''(0)\}.
\]
Zauważmy, że graniczne przypadki $c \downarrow 0$ i $c \uparrow \infty$ 
odpowiadają odpowiednio odbiciu i absorpcji w $0$. 
Jest to generator prawdopodobieństwa — dowód jest pozostawiony jako ćwiczenie.
Oto weryfikacja własności (d) w Definicji~\ref{}: 
Dla $g \in C_0[0, \infty)$ i $\lambda > 0$, 
musimy rozwiązać $f - \lambda L_c f = g$ dla $f \in \mathcal{D}(L_c)$. 
Niech $f_a \in \mathcal{D}(L_a)$ oraz $f_r \in \mathcal{D}(L_r)$ będą rozwiązaniami
\[
f_a - \lambda L_a f_a = g \quad \text{oraz} \quad f_r - \lambda L_r f_r = g.
\]
Ponieważ wszystkie trzy generatory są równe $\frac{1}{2} f''$ na swoich dziedzinach,
\[
f = \gamma f_a + (1 - \gamma) f_r
\]
jest wymaganym rozwiązaniem, pod warunkiem że $f'(0) = c f''(0)$. 
Ma to miejsce, gdy $\gamma$ spełnia
\[
\gamma f_a'(0) = c (1 - \gamma) f_r''(0).
\]
Aby znaleźć wartość $\gamma$, $f_a'(0)$ i $f_r''(0)$ muszą mieć ten sam znak. 
Aby to sprawdzić, rozważmy $h = f_a - f_r$. 
Wtedy $h - \frac{\lambda}{2} h'' \equiv 0$, więc, 
ponieważ $h$ jest ograniczone,
\[
	h(x) = h(0) e^{-x \sqrt{2/\lambda}}.
\]
Wynika z tego, że
\[
	f_a'(0) = -\sqrt{2/\lambda} h(0)
\]
oraz
\[
f_r''(0) = -\left(\frac{2}{\lambda}\right) h(0),
\]
więc mają ten sam znak, i
\[
\gamma = \frac{2c}{2c + \sqrt{2\lambda}}.
\]
Aby powiedzieć coś o zachowaniu tego procesu, gdy odwiedza 0, napiszmy

\[
f = \alpha U_c(\alpha) g,
\]
gdzie $\alpha = \lambda^{-1}$, a $U_c$ jest rozwiązaniem dla procesu $X_c(t)$ z generatorem $L_c$. 
Można to zapisać jako
\[
f(x) = \frac{2c f_a(x) + \sqrt{2\lambda} f_r(x)}{2c + \sqrt{2\lambda}} = \alpha \int_{0}^{\infty} e^{-\alpha t} \mathbf{E}_x 
g(X_c(t)) \,\mathrm{d}t.
\]
Zastosujmy tę tożsamość do ciągu funkcji $g$, które są nieujemne i rosną do $1_{(0, \infty)}$. 
Odpowiadające im $f$, $f_a$, oraz $f_r$ rosną odpowiednio do
\[
	\alpha \int_{0}^{\infty} e^{-\alpha t} \mathbf{P}_x(X_c(t) > 0) \,\mathrm{d}t, \quad 
	\alpha \int_{0}^{\infty} e^{-\alpha t} \mathbf{P}_x(X_a(t) > 0) \,\mathrm{d}t,
\]
oraz $1$. 
Biorąc $x = 0$, otrzymujemy
\[
	\mathbf{E}_0 \int_{0}^{\infty} \alpha e^{-\alpha t} 1_{\{X_c(t) > 0\}} \,\mathrm{d}t = 
	\frac{1}{1 + c\sqrt{2\alpha}}.
\]
Zatem miara Lebesgue'a zbioru $\{t \ge 0 : X_c(t) = 0\}$ jest dodatnia, w przeciwieństwie do przypadku procesu odbijanego, który odpowiada $c = 0$.
\end{pd}


\end{document}
